%----------------------------------------------------------
\subsection{System Model}

\begin{equation} \label{eq:system_equation}
P_{obj} = 
	\begin{pmatrix}
	x \\
	y \\
	z \\
	\end{pmatrix}
=
	\begin{pmatrix}
	c_x \\
	c_y \\
	c_z \\
	\end{pmatrix}
+
	\begin{pmatrix}
	r_{11} & r_{12} & r_{13} \\
	r_{21} & r_{22} & r_{23} \\
	r_{31} & r_{32} & r_{33} \\
	\end{pmatrix}
*
	\begin{pmatrix}
	x_c \\
	y_c \\
	z_c \\
	\end{pmatrix}
\end{equation}

%----------------------------------------------------------
\subsection{Camera Model}
	The model of the camera and its equations had been described on section 2.4. Matching Algorithm.
%----------------------------------------------------------
\subsection{Extended Kalman Filter}

The Kalman Filter is an iterative algorithm that obtain an estimation of a lineal dynamic system when sensors has white noise. In this case, the system is non-linear so it's used a variant called Extended Kalman Filter \cite{GabrielTerejanu} (or EKF ). To apply the algorithm it's necessary to use a linealization of the known system described previously.

Consider the following nonlinear system described by the difference equation and the observation model with additive noise: \\

\begin{gather}
x_k = f(x_{k-1}) + w_{k-1} \\
z_k = h(x_k) + v_k
\end{gather}

Been $x_k$ the state of the system, $z_k$ the observation state, and $f(�) \& h(�)$ the functions of both systems with their noise $ w_k ; v_k$. \\
Every step of the EKF consist on two substeps called "Predictor" and "Corrector":

\begin{itemize}
   \item Predictor Step
		\begin{gather}
			x_k^{f} \approx f(x_{k-1}^{a}) \\
			P_k^{f} = J_f(x_{k-1}^{a}) P_{k-1} J_f^{T}(x_{k-1}^{a}) + Q_{k-1}
		\end{gather}

	\item CorrectorStep
		\begin{gather}
			P_k = (I - K_k J_h(x_k^{f}))P_k^{f} \\
			K_k = P_k^{f} J^{T}_h(x_k^{f}) (J_h(x_k^{f}) P_k^{f} J_h^{T}(x_k ^{f}) + R_k)^{-1} \\
			x_k^{a} \approx x_k^{f} + K_k (z_k - h(x_k^{f}))
		\end{gather}
\end{itemize}

In the previous ecuations, $J_h$ and $J_f$ are the Jacobian matrix of the system and observator. They are can be computed basing on the system and camera model as: \\

\[J_h =  
	\begin{pmatrix}
		\frac{\partial h_1}{\partial x_1} & \dots & \frac{\partial h_1}{\partial x_n} \\
		\vdots & \ddots & \vdots \\
		\frac{\partial h_m}{\partial x_1} & \dots & \frac{\partial h_m}{\partial x_n}
	\end{pmatrix}
\]

And so on with the $ f(�)$ function for the $J_f$ jacobian. Following paragraphs decribes the procedure of acquisition of both jacobians for this specific situation. \\

\begin{equation} \label{eq:observation_equation}
z_k =
	\begin{pmatrix}
		x_{img} \\
		y_{img}
	\end{pmatrix}
\stackrel{\ref{eq:pinhole_cam_eq}}{=}
	\begin{pmatrix}
		f*\frac{y_c}{x_c} \\
		f*\frac{z_c}{x_c}
	\end{pmatrix}
\end{equation}

Getting \ref{eq:system_equation} and isolating the position vector of the target in the camera's base, then is possible to express the observation state in term of real system state.

\begin{equation}
\ref{eq:system_equation} \Rightarrow 
	\begin{pmatrix}
		x_c \\
		y_c \\
		z_c 
	\end{pmatrix}
=
	\begin{pmatrix}
		r_{11} & r_{21} & r_{31} \\
		r_{12} & r_{22} & r_{32} \\
		r_{13} & r_{23} & r_{33}
	\end{pmatrix}
*
	\begin{pmatrix}
		x - c_x \\
		y - c_y \\
		z - c_z
	\end{pmatrix}
\end{equation}

So, individually:

\begin{equation}
\left\{ 
	\begin{split}
		x_c = r_{11}(x-c_x) + r_{21}(y-c_y) + r_{31}(z-c_z) \\
		y_c = r_{12}(x-c_x) + r_{22}(y-c_y) + r_{32}(z-c_z) \\
		z_c = r_{13}(x-c_x) + r_{23}(y-c_y) + r_{33}(z-c_z) 
	\end{split} \right.
\end{equation}

Introducing this in equation \ref{eq:observation_equation}:

\begin{equation}
z_k =
	\begin{pmatrix}
		f�\frac{r_{12}(x-c_x) + r_{22}(y-c_y) + r_{32}(z-c_z)}{r_{11}(x-c_x) + r_{21}(y-c_y) + r_{31}(z-c_z)} \\
		f�\frac{r_{13}(x-c_x) + r_{23}(y-c_y) + r_{33}(z-c_z)}{r_{11}(x-c_x) + r_{21}(y-c_y) + r_{31}(z-c_z)}
	\end{pmatrix}
\end{equation}
